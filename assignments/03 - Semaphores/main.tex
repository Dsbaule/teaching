\documentclass[11pt]{article}
\usepackage[utf8]{inputenc}
\usepackage[brazilian]{babel}

\usepackage[parfill]{parskip}

% Identação de blocos.
\usepackage{scrextend}

% URLs.
\usepackage[hidelinks,colorlinks=true,urlcolor=blue]{hyperref}

\title{Semáforos}

\author{Pedro H. Penna, João Caram e Henrique C. Freitas\\[0.3em]
\small Pontifícia Universidade Católica de Minas Gerais}
\date{}

\begin{document}

\maketitle

\begin{abstract}

\noindent Mecanismos de sincronização inter-processos possibilitam que processos acessem recursos compartilhados de maneira segura. Nesse projeto você implementará a abstração de semáforos no \textit{kernel}.

\end{abstract}

\subsubsection*{Fundamentação Teórica}

Um semáforo é uma primitiva de sincronização inter-processos proposta por Edsger Dijkstra em 1965. Nesse mecanismo, uma variável especial protegida, do tipo abstrato semáforo, possibilita o acesso exclusivo a recursos compartilhados por meio de quatro operações básicas:

\begin{itemize}
	\item \texttt{create(n)}: cria um semáforo semáforo e o inicializa com o valor \texttt{n}.
	\item \texttt{down()}: testa o valor do semáforo. Se esse valor for maior que zero, ele é decrementado e o processo continua sua execução normalmente. Caso contrário, o processo é bloqueado.
	\item \texttt{up()}: testa se o valor do semáforo é zero. Em caso afirmativo e existir algum processo bloqueado no semáforo, o processo é desbloqueado. Caso contrário o valor do semáforo é incrementado.
	\item \texttt{destroy()}: destrói o semáforo.
\end{itemize}

As operações \texttt{down()} e \texttt{up()} são atômicas, isto é, enquanto um processo executa uma dessas operações, nenhum outro pode executar outra operação sobre o mesmo semáforo. Assim, o acesso exclusivo a um recurso compartilhado é garantido.

\subsubsection*{Descrição do Projeto}

O Nanvix não oferece nenhuma primitiva de sincronização inter-processos. Processos podem manipular regiões de memória compartilhada e arquivos, mas sem a garantia de que não irão atrapalhar a computação de outros processos que acessam os mesmos recursos compartilhados.

Para resolver esse problema, você deverá implementar abstração de semáforos em \textit{kernel}. Essas operações devem ser exportadas como chamadas de sistema e devem seguir as seguintes especificações.

\texttt{int semget(unsigned key)}

	\begin{addmargin}[2em]{0em}

	A função \texttt{semget} permite que um processo use um semáforo associado a uma chave (\texttt{key}).
	Caso nenhum semáforo esteja associado a \texttt{key}, um novo semáforo deve ser criado.

	Em caso de conclusão com êxito, a função deve retornar o identificador do semáforo associado a \texttt{key}.
	Em caso de erro, -1 deve ser retornado.
	
	\end{addmargin}

\texttt{int semctl(int semid, int cmd, int val);}

	\begin{addmargin}[2em]{0em}

	A função \texttt{semctl()} permite um série de operações de controle no semáforo identificado por \texttt{semid}.
	Essas operações são especificadas por \texttt{cmd} e podem ser:
	\texttt{GETVAL}, retorna o valor corrente do semáforo;
	\texttt{SETVAL}, define o valor do semáforo como sendo \texttt{val}; e
	\texttt{IPC\_RMID}, desvincula o semáforo do processo corrente e o destrói caso não esteja mais sendo usado.

	O valor de retorno da função depende de \texttt{cmd}. 
	Caso esse seja \texttt{GETVAL}, o valor corrente do semáforo é retornado.
	Nos demais casos de conclusão com êxito, 0 deve ser retornado.
	Em caso de erro -1 deve ser retornado retornado.

	\end{addmargin}
	
\texttt{int semop(int semid, int op);}

	\begin{addmargin}[2em]{0em}

	A função \texttt{semop} permite operações atômicas de incremento/decremento no semáforo identificado por \texttt{semid}.	
	Um valor negativo para \texttt{op} especifica um operação \texttt{down()} e um valor positivo uma operação \texttt{up()}.
	
	A função deve retornar 0 em uma conclusão com êxito, ou então -1 em caso de erro.
	
	\end{addmargin}
	
Junto com o projeto você deverá entregar uma pequena descrição sobre a solução implementada. Programas-exemplo serão fornecidos para que você teste sua implementação.

\subsubsection*{Por Onde Começar?}

Nesse projeto os seguintes arquivos serão relevantes:

\begin{itemize}
    \item \texttt{include/nanvix/syscall.h}: IDs das chamadas de sistema.
    \item \texttt{kernel/pm/sleep.c}: implementação das funções \texttt{sleep()} e \texttt{wakeup()}.
    \item \texttt{kernel/sys/syscalls.c}: tabela de chamadas de sistema.
    \item \texttt{include/sys/sem.h}: interface de usuário para manipulação de semáforos.
    \item \texttt{lib/libc/sys/sem/semget.c}: função de biblioteca \texttt{semget()}.
    \item \texttt{lib/libc/sys/sem/semctl.c}: função de biblioteca \texttt{semctl()}.
    \item \texttt{lib/libc/sys/sem/semop.c}: função de biblioteca \texttt{semop()}.
\end{itemize}

No entanto, você também deverá criar alguns outros arquivos:

\begin{itemize}
    \item \texttt{kernel/sys/sem.c}: implementação da abstração de semáforos.
    \item \texttt{kernel/sys/sem/semget.c}: chamada de sistema \texttt{sys\_semget()}.
    \item \texttt{kernel/sys/sem/semctl.c}: chamada de sistema \texttt{sys\_semctl()}.
    \item \texttt{kernel/sys/sem/semop.c}: chamada de sistema \texttt{sys\_semop()}.
\end{itemize}

\subsubsection*{Indo Além}

especificações para manipulação de semáforos nesse projeto consistem em uma simplificação da versão estabelecida pelo padrão Posix, disponível em \url{pubs.opengroup.org/stage7tc1/basedefs/sys\_sem.h.html} 

Caso você deseje implementar uma interface mais robusta, que esteja mais conforme o Posix, você será recompensado pelo seu esforço. Procure o professor da disciplina para discutir o assunto em maiores detalhes. Além disso, o grupo que possuir a implementação mais robusta será convidado a submeter um \textit{pull} no repositório de desenvolvimento do Nanvix.

\end{document}

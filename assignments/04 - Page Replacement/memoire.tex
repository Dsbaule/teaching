\documentclass[a4paper,11pt]{article}

\usepackage[utf8]{inputenc}             
\usepackage[T1]{fontenc}              

\usepackage{fullpage}
\usepackage{array,multirow,bigdelim}
\usepackage{charter}
\usepackage{paralist}
\usepackage{xcolor}

\usepackage[francais]{babel} 
\usepackage{listings}
\usepackage{url}
\usepackage{authblk}

\usepackage{graphicx}

\usepackage{mathpazo}

\usepackage{tikz}
\usetikzlibrary{shapes.multipart, calc, arrows}

\tikzstyle{bplus}=[rectangle split, rectangle split horizontal,rectangle split ignore empty parts,draw, fill=white]
\tikzstyle{every node}=[bplus]
\tikzstyle{level 1}=[sibling distance=90mm]
\tikzstyle{level 2}=[sibling distance=45mm]
\tikzstyle{level 3}=[sibling distance=30mm]
\usetikzlibrary{arrows}
\usetikzlibrary{chains}


\newcommand\eg{$=$}
\newcommand\pl{$+$}
\newcommand\mn{$-$}
\newcommand\sep{;~~~~}


\title {Architectures des Systèmes (AS)\\
Polytech Grenoble (RICM4)\\
\vspace {0.3cm}
Stratégie de remplacement de page
\author {Pedro Penna$^{1,2}$, Henrique Freitas$^{1}$, Marcio Castro$^{2}$, Jean-François Méhaut$^{3}$
  \\
  $^{1}$ Pontifical University of Minas Gerais (PUC Minas), Belo Horizonte, Brazil\\
  $^{2}$ Federal University of Santa Catarina (UFSC), Florianapolis, Brazil\\
  $^{3}$ Université Grenoble Alpes (UGA), Polytech Grenoble, France
  }
\date {18 Janvier 2017}
}

\begin{document}

\maketitle

L'espace de mémoire d'un ordinateur est organisé de manière hiérarchique avec différents niveaux
de cache (L1, L2 et L3), des registres et la mémoire centrale (DRAM). Le système
d'exploitation tient un rôle de gestionnaire. Plus précisément, il aura à allouer un espace
de mémoire physique pour le code et les données des programmes/processus. L'approche utilisée
par les systèmes d'exploitation repose sur la pagination.

Avec cette approche, une des fonctionnalités principales est de maintenir dans la mémoire physique
les pages les plus utilisées et de placer sur les disques les pages qui ne sont pas utilisées. Dans
le système Nanvix, ce travail est effectué par le composant de remplacement de page.Au cours de cette
étape, vous proposerez des améliorations au composant de remplacement de page qui vous est fourni.


\section*{Contexte}

Le principal objectif pour le système d'exploitation est de donner à
l'utilisateur l'illusion qu'il dispose d'un espace mémoire infiniment
grand, rapide et non volatile. C'est ainsi que la mémoire de
l'ordinateur est organisé hiérarchiquement autour d'un espace mémoire
volatile (registre, cache, mémoire physique DRAM) et d'un espace
mémoire non volatile (disques).  Avec cette approche, les espaces de
mémoire rapide et de taille réduite sont placés près du processeeur,
alors que les espaces mémoire de grande taile et moins rapide sont
plus éloignées du processeur.

Avec cette hiérarchie, le rôle du système d'exploitation doit être
celui d'un gestionnaire, qui travaille de manière efficace (décisions
rapides) et transparente. Pour atteindre ces objectifs, les systèmes
d'exploitation modernes sur une technique hybride
(matérielle/logicielle) appelée mémoire virtuelle ({\em virtual
  memory}).

L'approche de la mémoire virtuelle repose sur le mécanisme de
pagination. L'espace d'adressage d'un processus est découpé dans des
entités de taille égale qu'on appelle des pages. Les pages de la
mémoire virtuelle sont placées dans des cadres de mémoire physique
qu'on appelle en anglais {\em page frames} ou en Français des cadres
de mémoire physique. Les pages de la mémoire virtuelle d'un processus
sont placées, soit en mémoire centrale physique, soit sur un espace de
mémoire secondaire comme un disque.  De cette manière, un ensemble de
processus peut partager la mémoire centrale physique, même si toutes
les pages ne tiennent pas toutes en mémoire au même moment.

Pour maintenir cette illusion de l'espace mémoire infini avec les
pages de mémoire virtuelle et les cadres de mémoire physique, le
travail du système d'exploitation consiste à maintenir dans la mémoire
physique (dans les cadres de mémoire physique) toutes les pages qui
sont fréquemment utilisées et de stocker sur le disque celles qui ne
le sont pas. Cette gestion des pages de mémoire virtuelle a un grand
impact sur les performances. Il s'agit en particulier d'essayer de
minimiser les défauts de page et par conséquent minimiser les
opérations de lecture ou d'écriture disque qui sont particulièrement
coûteuses.


\section*{Description du travail}

Le composant de remplacement de page du système Nanvix repose sur la
stratégie du premier entré premier sorti ({\em Fisrt In First
  Out}). Quand un défaut de page survient, ({\bf et qu'il n'y a plus
  de place en mémoire physique?}), le système choisit comme page
vicitime pour le processus la plus vieille page qui soit en mémoire
physique. Cette page victime sera écrite sur le disque et la page
manquante sera placée dans le cadre de mémoire physique qui vient de
se libérer. Cependant, cette stratégie a de mauvaises performances,
plus précisément quand le processus présente des schémas d'accès
réguliers aux données, ce qui est particulièrement vrai pour le
programme de multiplication de matrices.

Si on reprend ce programme de multiplication de deux matrices qui ne
tiennent pas complètement dans la mémoire physique de
l'ordinateur. Dans ce cas, la meilleure stratégie n'est pas de
conserver en mémoire les pages les plus récemment utilisées, mais de
maintenir en mémoire celles qui appartiennet à un ensemble de travail
({\em working set}) du programme, que sont les pages qui contiennent
les lignes et les colonnes courantes qui doivent être multipliées.

Pour résoudre ce problème, vous allez modifier l'algorithme de
remplacement de page qui est fourni avec Nanvix. Vous proposerez un
nouvel algorithme qui offre de meilleures performances. Vous
présenterez une analyse de performance quantitative de votre
solution. Vous utiliserez le programme de multiplication de matrices
qui est fourni avec le système. N'oubliez pas que des algorithmes
basés sur le {\em working set} ou ceux basés sur le vieillissement
({\em aging}) conduisent à de bonnes performances.

\section*{Conseils}

L'implémentation de la gestion mémoire se trouve dans le répertoire {\tt kernel/mm}.
Il y a plusieurs fichiers:

\begin{itemize}
\item {\tt mm.c} initialise le sous-système de gestion mémoire.

\item {\tt paging.c} contient le système de pagination ({\em swapping}).

\item {\tt region.c} contient la gestion des régions.   
\end{itemize}

Pour cette étape, vous travaillerez principalement sur le fichier {\tt paging.c}.
Plus précisément dans la fonction {\tt allocf} qui implémente l'algorithme de remplacement
de page. Le programme source du code de  multiplication de matrice se trouve dans le
fichier {\tt src/sbin/test/test.c}


\end{document}

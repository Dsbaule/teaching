\documentclass[11pt]{article}
\usepackage[utf8]{inputenc}
\usepackage[brazilian]{babel}

\usepackage{subfigure}
\usepackage{indentfirst}
\usepackage{graphicx}
\usepackage{xcolor}
\usepackage[cm]{fullpage}
\usepackage{hyperref}

\usepackage{enumitem}

\newlist{legal}{enumerate}{10}
\setlist[legal]{label*=\arabic*.}

\usepackage{listings}
\lstset{
  basicstyle=\footnotesize\ttfamily,
}

\newcommand*{\alert}[1]{\vspace{0.4cm}\colorbox{gray!60!white}{\parbox{0.92\linewidth}{{\centering \textbf{IMPORTANTE!}\\}#1}}\vspace{0.4cm}}

\title{Guia Básico do Nanvix: Desenvolvimento de Utilitários de Sistema}

\author{Fernando Jorge Mota e Márcio Castro\\[0.3em]
\small Universidade Federal de Santa Catarina}
\date{}

\hyphenation{tool-chain}

\usepackage{url}

\begin{document}

\maketitle

\section{Introdução}

O Nanvix possui um conjunto de utilitários básicos de sistema que são compilados juntamente com o \textit{kernel}, tais como \texttt{cat}, \texttt{echo}, \texttt{cp}, \texttt{ps}, \texttt{kill}, \texttt{touch}, entre outros. Porém, é possível você criar o seu próprio utilitário de forma que o mesmo fique disponível e acessível no terminal do Nanvix. Esse guia mostra os passos que você deve seguir para criar um utilitário ``hello world''.

\section{Criando um Utilitário ``Hello World''}

Você deverá realizar alguns poucos passos:

\begin{legal}
	\item Crie uma pasta em \texttt{src/ubin} com o nome do utilitário que você deseja criar. Chamaremos essa pasta de \texttt{hello} para os fins desse tutorial.
	\item Dentro dessa pasta, crie um arquivo com extensão ``\texttt{.c}''. O nome não tem restrições, mas usaremos, para fins ilustrativos, o nome \texttt{hello.c}.
	\item Nesse arquivo, coloque o seguinte código:

\begin{lstlisting}[language=c,numbers=none,frame=single]
#include <stdio.h>
int main() {
   printf("Hello World!\n");
   return 0;
}
\end{lstlisting}

	\item Agora, no arquivo \texttt{makefile} da pasta \texttt{src/ubin}, faça as seguintes modificações:

	\begin{legal}
		\item Inclua, ao final da seção que possui as linhas começando com \texttt{.PHONY}: uma linha com o seguinte valor:

\begin{lstlisting}[language=bash,numbers=none,frame=single]
.PHONY: hello
\end{lstlisting}

		\item Depois, na linha seguinte a que começa com \texttt{all:} inclua o nome do binário (\texttt{hello}):

\begin{lstlisting}[language=bash,numbers=none,frame=single]
# Builds everything.
all: cat chgrp chmod chown cp echo kill ln login ls mv nice pwd rm stat sync \
     touch tsh ps clear nim hello
\end{lstlisting}
	
	\item Inclua uma linha no final do arquivo com o seguinte conteúdo, de forma a permitir a compilação do binário com ligação  à \texttt{libc} do Nanvix:

\begin{lstlisting}[language=bash,numbers=none,frame=single]
# Builds hello.
hello:
   $(CC) $(CFLAGS) $(LDFLAGS) hello/*.c -o $(UBINDIR)/hello $(LIBDIR)/libc.a
\end{lstlisting}

	\item Por fim, inclua após a seção \texttt{clean:} a seguinte linha para apagar o binário gerado caso venhamos a usar \texttt{make clean}:

\begin{lstlisting}[language=bash,numbers=none,frame=single]
@rm -f $(UBINDIR)/hello
\end{lstlisting}
	\end{legal}
\end{legal}

Após essas modificações, você poderá compilar novamente o Nanvix (comando \texttt{make nanvix}) e gerar a imagem com o novo binário (comando \texttt{make image}). Depois, execute a máquina virtual usando o comando \texttt{bash tools/run/run.sh} e, no terminal do Nanvix, use o comando \texttt{hello} para executar o programa utilitário criado por você. A mensagem ``\texttt{Hello World!}'' deverá ser impressa na tela.

\alert{Observe que, por se tratar de uma \texttt{libc} limitada, nem todo código que funciona no Linux vai compilar e executar corretamente no Nanvix. Além disso, o binário gerado deverá ter um tamanho limitado para não ultrapassar o limite do tamanho do sistema de arquivos do Nanvix.}

\end{document}

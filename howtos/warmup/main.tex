\documentclass[11pt]{article}
\usepackage[utf8]{inputenc}
\usepackage[brazilian]{babel}

\usepackage{subfigure}
\usepackage{indentfirst}
\usepackage{graphicx}
\usepackage{xcolor}
\usepackage[cm]{fullpage}
\usepackage{hyperref}

\usepackage{enumitem}

\newlist{legal}{enumerate}{10}
\setlist[legal]{label*=\arabic*.}

\usepackage{listings}
\lstset{
  basicstyle=\footnotesize\ttfamily,
}

\newcommand*{\alert}[1]{\vspace{0.4cm}\colorbox{gray!60!white}{\parbox{0.92\linewidth}{{\centering \textbf{IMPORTANTE!}\\}#1}}\vspace{0.4cm}}

\title{Atividade Prática no Nanvix: Github Classroom Warm-up}

\author{Fernando Jorge Mota e Márcio Castro\\[0.3em]
\small Universidade Federal de Santa Catarina}
\date{}

\hyphenation{tool-chain}

\usepackage{url}

\begin{document}

\maketitle

\section{Introdução}

Antes de realizar as atividades propostas neste documento é necessário que você tenha algumas noções básicas sobre o Nanvix e o Github Classroom. Para isso, você deverá ler atentamente os seguintes documentos disponíveis na seção \textit{Nanvix} do Moodle:

\begin{enumerate}
	\item \textit{Instruções de Download, Uso e Compilação}: contém as informações básicas sobre como compilar e utilizar o Nanvix;
	\item \textit{Instruções sobre o Git e o Github Classroom}: contém as informações básicas sobre o Git e um passo à passo para começar a utilizar o Github Classroom.
\end{enumerate}

\section{Atividades Práticas}

Agora que você já leu os documentos mencionados anteriormente, vamos realizar algumas atividades práticas. Elas servirão de aquecimento para os trabalhos práticos da disciplina.

\begin{enumerate}
	\item Realize os passos descritos no documento \textit{Guia Básico do Nanvix: Desenvolvimento de Utilitários de Sistema}, disponível na seção \textit{Nanvix} do Moodle, para criar o programa \texttt{hello} no Nanvix e faça um teste do mesmo.
	\item Crie um novo programa utilitário no Nanvix denominado \texttt{fork1}. Quando executado no Nanvix, esse programa deverá criar um processo filho utilizando a chamada de sistema \texttt{fork()}. Ambos os processos pai e filho deverão imprimir na tela a frase \texttt{``Novo processo criado!''}. Você deverá utilizar apenas um \texttt{printf()} para isso.
	\item Crie um novo programa utilitário no Nanvix denominado \texttt{fork2}. Quando executado no Nanvix, esse programa deverá criar 4 processos filhos. Para cada filho criado, o processo pai deverá imprimir na tela \texttt{``Processo pai XX criou YY''}, onde \texttt{XX} é o PID do pai e \texttt{YY} o PID do filho. Além disso, os processos filhos deverão imprimir na tela \texttt{``Processo filho XX''}, onde \texttt{XX} é o PID do filho. Dica: utilize a função \texttt{getpid()} para retornar o PID do processo corrente.
\end{enumerate}

\end{document}
